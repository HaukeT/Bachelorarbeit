\lstset{language=C}

%%%%%%%%%%%%%%%%%%%%%%%%%%%%%%%%%%%%%%%%%%%%%%%%%%%%%%%%%%%%%%%%
%
% Kapitel: EINLEITUNG
%
%%%%%%%%%%%%%%%%%%%%%%%%%%%%%%%%%%%%%%%%%%%%%%%%%%%%%%%%%%%%%%%%

\chapter{Einleitung}
\section{Thema}
In dieser Bachelorarbeit geht es darum die versteckten Themen, in einem Patentdatensatz von General Motors, zu finden. Diese Themen sollen benannt und graphisch dargestellt werden, um herauszufinden welche Themengruppen es gibt und welche Patente zu einem oder mehreren Themen gehören. Außerdem soll die Entwicklung der Themen über die Jahre untersucht werden.

% 2. Problemstellung
\section{Motivation und Zielsetzung}
Uns standen noch nie so viele Informationen zur Verfügung wie heute und jeden Tag kommen neue hinzu. Wir durchsuchen schriftliche Informationen nach Stichwörtern, mit der Hilfe von Suchmaschinen. So lassen sich zu einem Thema schnell mehrere Texte finden.\\

Man beschreibt ein Thema aus Stichwörtern und sucht Texte, welche diese enthalten. Wenn man diese Suche umdreht funktioniert dies nicht mehr. Man hat einen Datensatz aus Texten und möchte alle darin enthaltenen Themen herausfinden. Intuitiv denkt man hier an den Titel aber der reicht nicht aus, um alle Themen eines Textes zu beschreiben. Allein der Titel dieser Arbeit verschweigt das Thema der Programmiersprache Python. Manche Texte haben Schlagworte aber hier verlässt man sich auf den Autor, die Richtigen zu wählen und sie werden nicht nach Relevanz gewichtet. Außerdem könnte man, mit dem Wissen über die Entwicklung der Patentthemen, Vermutungen über die Patentthemen der Zukunft anstellen.\\

Topic Modeling finde ich besonders interessant, weil man mit relativ geringem Aufwand große Mengen an Dokumenten untersuchen kann. Dadurch könnte man, speziell in diesem Fall, für die Konkurrenten von General Motors herausfinden worum es in den Patenten geht und in welche Richtung sich die Themen der Patente in Zukunft entwickeln könnten. Wodurch man General Motors bei der Anmeldung von neuen Patenten zuvorkommen und Lizenzgebühren verlangen könnte.\\

Also wie findet man in einem Textdatensatz die enthaltenen Themen und ihren zeitlichen Verlauf?\\

% 3. Zielsetzung
\section{Methodisches Vorgehen}
Um die versteckten Themen zu finden werden generative Wahrscheinlichkeitsmethoden benutzt. Eine Methode ist die Latent Dirichlet Allocation. \parencite[vgl.][]{Blei03latentdirichlet} Zuerst wird eine bestimmte Zahl an Themen festgelegt. Wörter die häufig gemeinsam vorkommen werden einem gemeinsamen Thema zugeordnet. Nachdem alle Wörter mindestens einem Thema zugeordnet wurden, wird der Vorgang für eine höhere Zahl an Themen wiederholt bis man genug Modelle hat, um sie zu vergleichen. Aus den Modellen wird das mit der höchsten Kohärenz ausgewählt. \parencite[vgl.][]{TopicCoherence}\\

Die wahrscheinlichsten Wörter eines Themas könnten lauten Ventil, Hydraulik und Flüssigkeit. Dieses Thema kann dann wiederum Texten zugeordnet werden. Mit dieser Methode lassen sich die Themen eines Datensatzes von hunderten Dokumenten viel schneller herausfinden, als es einem Menschen allein möglich wäre.\\

Am Beispiel des Patentdatensatzes von General Motors werde ich die Modelle des Online Latent Dirichlet Allocation Verfahrens \parencite[vgl.][]{Hoffman10onlinelearning} und des MALLET Verfahrens \parencite[vgl.][]{McCallumMALLET} auf Kohärenz vergleichen. Dabei werde ich auch die Kohärenzmaße C\_v und C\_umass vergleichen. \parencite[vgl.][]{TopicCoherence}\\

Der Patentdatensatz von General Motors umfasst über 1400 Patente für verschiedene Getriebearten und ist ausreichend groß um Topic Modeling zu betreiben.\\

Des weiteren werde ich mit dem dynamischen Latent Dirichlet Allocation Verfahren herausfinden wie sich die Themen des Datensatzes, entlang der zeitlichen Anmeldedaten der Patente, verändert haben. \parencite[vgl.][]{dynamicLDA} Besonders interessant wäre hier eine Veränderung des Themenschwerpunktes. Auch eine Vorhersage zu welchen Themen in Zukunft Patente angemeldet werden könnte möglich sein. Eine Vorhersage wäre für ein konkurrierendes Unternehmen hilfreich, um Patente vor General Motors anzumelden und Lizenzgebühren verlangen zu können.\\

Um diese Untersuchungen zu realisieren werde ich die Programmiersprache Python verwenden. Mit Hilfe der Programmbibliothek gensim \parencite[vgl.][]{rehurek_lrec} werde ich die Modelle erstellen und die Kohärenzen auswerten. Die Ergebnisse werde ich entsprechend ihrer Art visualisieren. Für die am häufigsten vorkommenden Themen werde ich LDAvis verwenden. \parencite[vgl.][]{sievert2014ldavis}













