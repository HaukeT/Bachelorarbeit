\lstset{language=C}

%%%%%%%%%%%%%%%%%%%%%%%%%%%%%%%%%%%%%%%%%%%%%%%%%%%%%%%%%%%%%%%%
%
% Kapitel: EINLEITUNG
%
%%%%%%%%%%%%%%%%%%%%%%%%%%%%%%%%%%%%%%%%%%%%%%%%%%%%%%%%%%%%%%%%

\chapter{Einleitung}
\section{Thema}
In dieser Bachelorarbeit geht es darum, die relevantesten Themen in einem großen Patentdatensatz von \gls{gm} zu finden. Dazu werden generativen Wahrscheinlichkeitsmethoden verwendet werden. Diese Themen sollen benannt und graphisch dargestellt werden. Außerdem gilt es herauszufinden welche Themengruppen vorhanden sind und welche Patente zu einem oder mehreren Themen gehören. Außerdem soll der Trend der Themen über die Jahre untersucht werden.

% 2. Problemstellung
\section{Motivation und Forschungsfragen}
Uns standen noch nie so viele Informationen zur Verfügung wie heute und jeden Tag kommen neue hinzu. Wir durchsuchen schriftliche Informationen nach Stichwörtern, mit der Hilfe von Suchmaschinen. So lassen sich zu einem Thema schnell mehrere Texte finden.

Man beschreibt ein Thema aus Stichwörtern und sucht Texte, welche diese enthalten. Wenn man diese Suche umkehrt, funktioniert sie nicht mehr. Man hat einen Datensatz aus Texten und möchte alle darin enthaltenen Themen herausfinden. Intuitiv denkt man hier an den Titel aber der reicht nicht aus, um alle Themen eines Textes zu beschreiben. Allein der Titel dieser Arbeit verschweigt das Thema des Preprocessings. Manche Texte haben Schlagworte aber dabei verlässt man sich auf den Autor, diese Richtigen zu wählen und sie werden nicht nach Relevanz gewichtet. Außerdem könnte man, mit dem Wissen über die Trends der Patentthemen, Vermutungen über die Patentthemen der Zukunft anstellen.

Topic Modeling ist besonders interessant, weil mit relativ geringem Aufwand große Mengen an Dokumenten untersucht werden können. Dadurch könnten, speziell in diesem Fall, die Konkurrenten von General Motors herausfinden, worum es in den Datensatz geht und in welche Richtung sich die Themen der Patente in Zukunft entwickeln könnten. Wodurch sie General Motors bei der Anmeldung neuer Patenten zuvorkommen und Lizenzgebühren verlangen könnten.

Die Forschungsfragen dieser Arbeit sind:
\begin{itemize}
	\item Welche relevanten Themen befinden sich in dem Datensatz?
	\item Wie sind diese Themen zu benennen?
	\item Welchen Gruppen gehören diese Themen an?
	\item Welche Trends haben diese Themen? 
\end{itemize}

% 3. Zielsetzung
\section{Methodisches Vorgehen}
Um die versteckten Themen zu finden, werden generative Wahrscheinlichkeitsmethoden benutzt. Eine Methode ist die Latent Dirichlet Allocation \parencite[vgl.][]{Blei03latentdirichlet}. Zuerst wird eine bestimmte Zahl an Themen festgelegt. Wörter die häufig gemeinsam vorkommen werden einem gemeinsamen Thema zugeordnet. Nachdem alle Wörter mindestens einem Thema zugeordnet wurden, wird der Vorgang für eine höhere Zahl an Themen wiederholt bis man genug Modelle hat, um sie zu vergleichen. Aus den Modellen wird das mit der höchsten Kohärenz ausgewählt \parencite[vgl.][]{TopicCoherence}.

Die wahrscheinlichsten Wörter eines Themas könnten lauten Ventil, Hydraulik und Flüssigkeit. Dieses Thema kann dann wiederum Texten zugeordnet werden. Mit dieser Methode lassen sich die Themen eines Datensatzes von tausenden Dokumenten viel schneller benennen, als es einem Menschen durch normales Lesen möglich wäre. Dazu wird das Online Latent Dirichlet Allocation Verfahren angewandt. \parencite[vgl.][]{Hoffman10onlinelearning} Als Kohärenzmaß wird C\_umass verwendet \parencite[vgl.][]{TopicCoherence}.

Der Patentdatensatz von General Motors umfasst 1410 Patente für verschiedene Getriebearten und ist ausreichend groß, um Topic Modeling zu betreiben. Um diese Untersuchungen zu realisieren, wird die Programmiersprache Python verwenden. Mit Hilfe der Programmbibliothek gensim werden die Modelle erstellen und die Kohärenzen ausgewertet \parencite[vgl.][]{rehurek_lrec}. Die Ergebnisse werden mit \gls{pyLDAvis} untersucht, um in einem qualitativen Verfahren Gruppen zu bilden und Themen zu benennen \parencite[vgl.][]{sievert2014ldavis}. Die Gruppen sollen mit \gls{hlda} bestätigt werden \parencite[vgl.][]{griffiths2004hierarchical}.

Des weiteren soll mit dem \gls{dlda} Verfahren herausgefunden werden wie sich die Themen des Datensatzes, entlang der zeitlichen Anmeldedaten der Patente, verändert haben \parencite[vgl.][]{dynamicLDA}. Eine Vorhersage wäre für ein konkurrierendes Unternehmen hilfreich, um Patente vor General Motors anzumelden und Lizenzgebühren verlangen zu können.















