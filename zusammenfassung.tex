\chapter{Zusammenfassung und Ausblick}




\section{Zusammenfassung}

Das Ziel dieser Arbeit war herauszufinden welche relevanten Themen sich in dem Patentdatensatz von \gls{gm} befinden, sie zu benennen und darzustellen wie sie sich über einen Zeitraum von 14 Jahren entwickelt haben.

Der Datensatz wurde mit einem Preprocessing zu Dokumenten mit einzelnen Termen vereinfacht. Die Terme wurden mit \gls{lda} einer möglichst optimalen Zahl an Themen zugeordnet. Die größten und relevantesten Themen wurden mit einem qualitativen Verfahren benannt und gruppiert. Dieses Verfahren wurde durch das Programm \gls{pyLDAvis} unterstützt. Die Gruppierungen wurden mit der alternativen Methode \gls{hlda} größtenteils bestätigt. Die Trends der relevantesten Themen und größten Gruppen wurden mithilfe von \gls{dlda} gefunden und dargestellt.

Die Ergebnisse dieser Arbeit ermöglichen es jedem die Hauptthemen und die meisten Unterthemen dieses Patentdatensatzes zu erfassen ohne ein einziges Patent lesen zu müssen. Außerdem ermöglichen die Ergebnisse das Filtern des Datensatzes nach bestimmten Themengruppen, Themen, Patenten und Termen. So könnten beispielsweise die relevantesten Patente für ein beliebiges Thema angezeigt werden. Des weiteren kann der Trend aller Terme visualisiert werden.



\section{Ausblick}

Wie in der Diskussion beschrieben würde es sich lohnen am Preprocessing der Bigramme zu arbeiten. Es wäre möglich das sich die Themen breiter verteilen, wenn mit den Erkenntnissen aus den Unigrammmen Terme aufgespalten oder zusammengelegt werden würden. Terme wie \emph{dualclutch} könnten in \emph{dual} \emph{clutch} aufegspalten werden und \emph{binary clutch} sollte zu \emph{dual clutch} werden. So könnten Bigramme wie \emph{clutch} \emph{dualclutch} vermieden werden. Dadurch sollte es beispielsweise nur noch ein Thema \emph{dual clutch} geben und nicht mehr das redundante Thema \emph{clutch dualclutch}. Mit dem vermeiden von Synonymen würde das Thema \emph{binary clutch} ebenfalls wegfallen.

Es wäre gut die Trends der Cluster und Themen direkt mit \gls{dlda} zu messen und nicht nur die Trends der Terme. Die gewählten Terme beschreiben die Cluster und Themen zwar gut und grenzen sie voneinander ab aber die Trends der Themen direkt zu messen wäre eine noch genauere Methode. 







